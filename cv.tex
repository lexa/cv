\documentclass[11pt,a4paper]{moderncv}
\usepackage{polyglossia}
\usepackage{graphicx}

\newif\ifenglish
\newif\ifrussian
\newcommand{\ml}[2]{\ifenglish#1\else#2\fi}

\directlua {
  local lang = os.getenv('CV_LANGUAGE')
  tex.print("\string\\setdefaultlanguage{" .. lang .. "}")
  tex.print("\string\\" .. lang .. "true")
}

\moderncvstyle{casual}
\moderncvcolor{blue}
\name{\ml{Aleksei}{Алексей}}{\ml{Fedotov}{Федотов}}
\title{\ml{Resume}{Резюме}}
\phone[mobile]{+49~(160)~910~84~853}
\email{aleksei@fedotov.email}
\social[linkedin]{aleksei-fedotov}
\setmainfont{CMU Bright}

\begin{document}
\makecvtitle
\setlength{\hintscolumnwidth}{0.20\textwidth}

% useful macroses
\def \YAR {\ml{Yaroslavl, Russia}{г. Ярославль}}
\def \MSK {\ml{Moscow}{Москва}}
\def \UNIV {\ml{Yaroslavl State University}{ЯРгУ им. П.Г. Демидова}}
\def \Samsung {\ml{Samsung R\&D Russia}{Исследовательский центр самсунг}}
\def \Samsunglogo {\\\includegraphics[width=20pt]{samsung.png}}
\def \OpSy {OpenSynergy GmbH}
\def \BER {Berlin, Germany}

\section{\ml{Professional skills}{Ключевые навыки}}
\begin{itemize}
\item \ml{Expertise with C language and understanding of ARM assembly language}{Знание языка C и понимание ARM assembly language}
\item \ml{Development in Linux environment i.e. experience with GCC, Make, Bash, cross compilation}{Опыт в разработке в Linux окружении с использованием GDB, GCC, Make, Bash в том числе и для кросс-компиляции}
\item \ml{General knowledge about IP networks and transport layer protocols}{Опыт админинистрирования IP сетей, знание сетевых протоколов. } TCP, UDP, ICMP, IPv4, IPv6, DHCP, ARP, VLAN(802.1Q), OSPF, BGP, STP
\item \ml{Maintaining and configuring network equipment, IP routers, Ethernet switches, Linux/FreeBSD servers}{Поддержка и настройка сетевого оборудования, IP роутеров, Ethernet коммутаторов, Linux/FreeBSD серверов}
\item \ml{Driver development for Linux kernel}{Написание модулей ядра Linux}
\item Debugging of use-space and kernel space, using GDB, JTAG Lauterbach Trace32.
\item \ml{Experience with version control systems Git, Perforce}{Умею пользоваться системами контроля версий Git, Perforce}
\item \ml{Working in a big distributed team, use of code review systems like Gerrit and Helix SWARM}{Имею опыт разработки в коллективе, использования код-ревью}
\item Using Yocto for development of embedded Linux distributions.
\end{itemize}

\ml{My interests lies in the following areas: System design, SW architecture, SW development, Research}{}

\section{\ml{Technical skills}{Технические навыки}}
\cvitem{\ml{Languages}{Языки}}{C, Bash, Python(basic), GNU Make, C++(basic knowledge), ARM Assembler Language}
\cvitem{VCS}{Git, Perforce}
\cvitem{\ml{OS}{ОС}}{GNU\textbackslash Linux, Android, Tizen}
\cvitem{Network}{Ethernet, VLAN(802.1Q), TCP, UDP, ARP, ICMP, STP, IPv4, IPv6, DHCP, OSPF, BGP}
\cvitem{\ml{Other}{Прочее}}{GDB, \LaTeX, Valgrind, Emacs, Doxygen, Yocto, low-level Android}

\pagebreak
\section{\ml{Experience}{Опыт работы}}
\cventry{2017--\ml{present}{Настощий момент}}{Software Engineer}{\OpSy}{\BER}{} {
  Development of Hypervisor for Automotive solutions.\newline
  \begin{itemize}
  \item Development of Yocto based Linux distribution for Automotive application.
  \item Porting Linux and Android kernel to a new boards.
  \end{itemize}
  }

\cventry{2013--2017}{\ml{Lead Software Engineer}{Ведущий Инженер-разработчик}}{\Samsung}{\MSK}{}{
  \ml{Develop secure embedded operating system for working under ARM TrustZone technology.}{Разработка защищенной POSIX-подобной ОС для работы на процессорах ARM.} \newline
  \begin{itemize}
  \item \ml{Development of SDK based on GNU tools (binutils, GCC)}{Разработка SDK на основе утилит GNU (binutils, GCC)}
  \item \ml{Development of automatic testing system for OS components}{Разработка системы автоматизированного тестирования компонентов ОС}
  \item \ml{Development of POSIX compatible system libraries (libc, libpthread) for embedded Unix-like OS}{Разработка POSIX-совместимых системных библиотек (libc, libpthread)}
  \item \ml{Development of native libraries and applications for Android and Tizen}{Разработка низкоуровневых (native) библиотек и приложений для Android и Tizen}
  \item \ml{Porting of existing code on 64bit platform}{Портирование существующего кода на 64битную платформу}
  \item \ml{Implementing GlobalPlatform TEE specification}{Разработка реализации GlobalPlatform TEE спецификации}
  \end{itemize}
}
\cventry{2012--2013}{\ml{Software Engineer}{Инженер-разработчик}}{\ml{YarSpecAlgo}{ЯрСпецАлгоритм}}{\YAR}{}{
    \ml{Design and development of software for VHF radio communicator}{Разработка ПО для управления УКВ радиостанцией}.\newline
  \begin{itemize}
  \item \ml{Development of client/server control software for embedded Linux (TCP)}{Разработка клиент-серверного ПО для встраиваемого Linux (TCP)}
  \item \ml{Development of real-time data transfer protocol over UDP}{Разработка протоколоа передачи данных поверх UDP в режиме рельного времени}
  \item \ml{Development of echo-compensation system on DSP Texas Instruments}{Разработка системы эхокомпенсации на DSP Texas Instruments}
  \end{itemize}
}

\cventry{2011--2013}{\ml{Network Engineer}{Сетевой Инженер}}{\ml{Yaroslavlteleset}{ЯрославльТелесеть}}{\YAR}{}{
    \ml{Maintaining of network infrastructure for Internet service provider}{Администрирование сетевой инфраструктуры провайдера}.\newline
  \begin{itemize}
  \item \ml{Configuration of network equipment (Cisco, HP, Edge-core)}{Настойка сетевого оборудования (Cisco, HP, Edge-core)}
  \item \ml{Design and maintenance of network infrastructure (OSPF, BGP, STP, VLAN)}{Поддержка сетевой инфраструктуры (OSPF, BGP, STP, VLAN)}
  \item \ml{Network monitoring and reliability assurance}{Мониторинг сети через Zabbix}
  \item \ml{Configuration and maintenance of Linux/FreeBSD servers}{Поддежка и настройка Application-серверов (CentOS, FreeBSD)}
  \end{itemize}
}

\section{\ml{Languages}{Знание языков}}
\cvitem{\ml{English}{Английский}}{Upper Intermediate}
\cvitem{\ml{Russian}{Русский}}{\ml{Native}{Родной}}

\section{\ml{Education}{Образование}}
% average grade (3*1 + 4*8 + 5*16)/(1+8+16) = 4.6
\cventry{2006--2010}{\ml{Bachelor}{Бакалавр}}{\UNIV}{\YAR}{\ml{Average grade}{Средний балл} 4.6}{}
% average grade (3*1 + 4*8 + 5*16 + 5*7)/(1+8+16+7) = 4.687
\cventry{2010--2012}{\ml{Master}{Магистр}}{\UNIV}{\YAR}{\ml{Average grade}{Средний балл} 4.7}{}

\end{document}
