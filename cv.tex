\documentclass[11pt,a4paper]{moderncv}
\usepackage{polyglossia}
\usepackage{graphicx}

\newif\ifenglish
\newif\ifrussian
\newcommand{\ml}[2]{\ifenglish#1\else#2\fi}

\directlua {
  local lang = os.getenv('CV_LANGUAGE')
  tex.print("\string\\setdefaultlanguage{" .. lang .. "}")
  tex.print("\string\\" .. lang .. "true")
}

\moderncvstyle{casual}
\moderncvcolor{blue}
\name{\ml{Aleksei}{Алексей}}{\ml{Fedotov}{Федотов}}
\title{\ml{Resume}{Резюме}}
\phone[mobile]{+7~(915)~984~59~77}
\email{a.fedotov@cfotr.com}
\social[linkedin]{aleksei-fedotov}
\setmainfont{CMU Bright}

\begin{document}
\makecvtitle
\setlength{\hintscolumnwidth}{0.20\textwidth}

% useful macroses
\def \YAR {\ml{Yaroslavl, Russia}{г. Ярославль}}
\def \MSK {\ml{Moscow}{Москва}}
\def \UNIV {\ml{Yaroslavl State University}{ЯРгУ им. П.Г. Демидова}}
\def \Samsung {\ml{Samsung R\&D Russia}{Исследовательский центр самсунг}}
\def \Samsunglogo {\\\includegraphics[width=20pt]{samsung.png}}

\section{\ml{Professional skills}{Ключевые навыки}}
\begin{itemize}
\item \ml{Experience in development and debugging of user space application for POSIX compatible OS}{Опыт в разработке и отладке приложений для POSIX совместимых ОС}
\item \ml{Experience in C development in Linux environment i.e. experience with GDB, GNU toolchain, cross compilation} {Опыт в разработке в Linux окружении с использованием GDB, GCC в том числе и для кросс-компиляции}
\item \ml{Driver development for Linux kernel}{Написание модулей ядра Linux}
\item \ml{General knowledge about ip networks and transport layer protocols like TCP,SCTP,UDP}{Знания о работе IP сетей и транспортных протоколов, таких как TCP, SCTP, UDP}
\end{itemize}

\ml{My interests lies in the following areas: System design, SW architecture, SW development, Research}{}
\section{\ml{Computer skills}{Технические навыки}}
\cvitem{\ml{Languages}{Языки}}{C, Bash, Python(basic), GNU Make, C++(basic), ARM Assembler Language}
\cvitem{VCS}{Git, Perforce}
\cvitem{\ml{OS}{ОС}}{GNU\textbackslash Linux, Android, Tizen}
\cvitem{\ml{Other}{Прочее}}{GDB, \LaTeX, Valgrind, Emacs, ARM, Doxygen}

\section{\ml{Experience}{Опыт работы}}
\cventry{2013--\ml{present}{Настощий момент}}{\ml{Lead Software Engineer}{Ведущий Инженер-разработчик}}{\Samsung}{\MSK}{}{
  \ml{Develop secure embedded operating system for working under ARM TrustZone technology.}{Разработка защищенной POSIX-подобной ОС для работы на процессорах ARM.} \newline
  \begin{itemize}
  \item \ml{Development of SDK based on GNU tools (binutils, GCC)}{Разработка SDK на основе утилит GNU (binutils, GCC)}
  \item \ml{Development of automatic testing system for OS components}{Разработка системы автоматизированного тестирования компонентов ОС}
  \item \ml{Development of POSIX compatible system libraries (libc, libpthread) for embedded Unix-like OS}{Разработка POSIX-совместимых системных библиотек (libc, libpthread)}
  \item \ml{Development of native libraries and applications for Android and Tizen}{Разработка низкоуровневых (native) библиотек и приложений для Android и Tizen}
  \item \ml{Porting of existing code on 64bit platform}{Портирование существующего кода на 64битную платформу}
  \item \ml{Implementing GlobalPlatform TEE specification}{Разработка реализации GlobalPlatform TEE спецификации}
  \end{itemize}
}
\cventry{2012--2013}{\ml{Software Engineer}{Инженер-разработчик}}{\ml{YarSpecAlgo}{ЯрСпецАлгоритм}}{\YAR}{}{
    \ml{Design and development of software for VHF radio communicator}{Разработка ПО для управления УКВ радиостанцией}.\newline
  \begin{itemize}
  \item \ml{Development of control software for embedded Linux.}{Разработка управляющего ПО для УКВ радиостанции основанной на Linux.}
  \item \ml{Development of echo-compensation system on DSP Texas Instruments.}{Разработка системы эхокомпенсации на DSP Texas Instruments.}
  \end{itemize}
}

\section{\ml{Languages}{Знание языков}}
\cvitem{\ml{English}{Английский}}{Upper Intermediate}
\cvitem{\ml{Russian}{Русский}}{\ml{Native}{Родной}}

\section{\ml{Education}{Образование}}
% average grade (3*1 + 4*8 + 5*16)/(1+8+16) = 4.6
\cventry{2006--2010}{\ml{Bachelor}{Бакалавр}}{\UNIV}{\YAR}{\ml{Average grade}{Средний балл} 4.6}{}
% average grade (3*1 + 4*8 + 5*16 + 5*7)/(1+8+16+7) = 4.687
\cventry{2010--2012}{\ml{Master}{Магистр}}{\UNIV}{\YAR}{\ml{Average grade}{Средний балл} 4.7}{}

\end{document}
